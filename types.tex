\begin{frame}
\frametitle{What is Parametricity}
\begin{block}{Philip Wadler \cite{wadler1989theorems} tells us:}
\begin{quotation}
Write down the definition of a polymorphic function on a piece of paper. Tell me its type, but be careful not to let me see the function's definition. I will tell you a theorem that the function satisfies.

The purpose of this paper is to explain the trick.
\end{quotation}
\end{block}
\end{frame}

\begin{frame}[fragile]
\frametitle{Types}
\begin{block}{first let's talk about types}
Suppose we encountered the following function definition:
\begin{lstlisting}
int add12(int)
\end{lstlisting}
\end{block}
\begin{itemize}
  \item<1-> by the type alone, there are {$({2^{32}})^{2^{32}}$} possible implementations
  \item<2-> but this is a significantly smaller number than \rotatebox{90}{8}
  \item<3-> \emph{Importantly, we know nothing more about this function from its type}
\end{itemize}
\end{frame}

\begin{frame}[fragile]
\frametitle{Types}
\framesubtitle{reading the code}
We might form a suspicion that \lstinline[style=scala]$add12$ adds twelve to its argument
\begin{lstlisting}[style=scala]
int `add12`(int)
\end{lstlisting}
\begin{tikzpicture}[remember picture,overlay]
\coordinate (aa) at ($(a1)+(7,2.0)$);
\node[note,draw,callout relative pointer={($(aa)-(12.8,-3.3)$)},right] at (aa) {\includegraphics[width=0.2\textwidth]{image/suspicion.jpg}};
\end{tikzpicture}
\end{frame}

\begin{frame}[fragile]
\frametitle{Types}
So we write some speculative tests to relieve our anxiety:
\begin{lstlisting}
add12(0)        = 12
add12(5)        = 17
add12(-5)       = 7
add12(223)      = 235
add12(5096)     = 5104
add12(2914578)  = 29145590
add12(-2914578) = -29145566
\end{lstlisting}
And pat ourselves on the back, concluding, yes, this function adds twelve to its argument
\end{frame}

\begin{frame}[fragile]
\frametitle{Types}
\framesubtitle{Woops!}
\begin{block}{but we are never sure}
\begin{lstlisting}
def add12(n: Int): Int =
  if(n < 8000000) n + 12
  else n * 7
\end{lstlisting}
\end{block}
We need to narrow down the potential propositions about what this function does not do.
\end{frame}

\begin{frame}
\frametitle{Monomorphic}
\begin{block}{this reasoning is difficult because}
our types are \emph{monomorphic}
\end{block}
\end{frame}

\begin{frame}[fragile]
\frametitle{Types}
\begin{block}{another monomorphic example}
\begin{lstlisting}
List<int> function(List<int>)
\end{lstlisting}
\end{block}
\begin{itemize}
  \item<1-> adds 17 to every 11th element?
  \item<2-> drops every prime number?
\end{itemize}
\end{frame}

\begin{frame}[fragile]
\frametitle{Parametricity}
\begin{block}{a polymorphic example}
\begin{lstlisting}
<A> List<A> function(List<A>)
\end{lstlisting}
\end{block}
\begin{itemize}
  \item<1-> this function returns elements in a list that always appear in the argument
  \item<1-> \textbf{or it would not have compiled}
  \item<2-> Convince yourself of this. Commit to this statement.
\end{itemize}
\end{frame}

\begin{frame}[fragile]
\frametitle{Parametricity}
\begin{block}{To clarify}
\begin{lstlisting}
[a] -> [a]
\end{lstlisting}
\end{block}
For any function with this type, in the entire universe, every element in the return value appears in the argument value
\end{frame}

\begin{frame}[fragile]
\frametitle{Parametricity}
\begin{block}{the goal}
\begin{itemize}
  \item<1-> a significant number of possible things that this function does are eliminated, by no expenditure of effort
  \item<2-> theorems about this function can be reliably constructed
\end{itemize}
\end{block}
\end{frame}

\begin{frame}[fragile]
\frametitle{Parametricity}
\begin{block}{here is another one \emph{(Haskell)}}
\begin{lstlisting}
Maybe a -> [a]
\end{lstlisting}
\end{block}
\begin{itemize}
  \item<1-> all the \lstinline{a} values in the returned list are the same value
  \item<2-> if the argument is \lstinline{Nothing} then \lstinline{[]} is returned \emph{(no \lstinline{a} values)}
  \item<3-> the \lstinline{a} values in the return result is the one in \lstinline{Just}.
\end{itemize}
\end{frame}

\begin{frame}[fragile]
\frametitle{Parametricity}
\begin{center}
\begin{lstlisting}
Maybe a -> [a]
\end{lstlisting}
\end{center}
\begin{block}{I know these theorems}
\textbf{because it compiles}
\end{block}
\end{frame}

\begin{frame}[fragile]
\frametitle{Parametricity}
\begin{block}{one more}
\begin{lstlisting}
Functor f =>
b -> f a -> f b
\end{lstlisting}
\end{block}
\end{frame}

\begin{frame}[fragile]
\frametitle{Parametricity}
\begin{block}{one more}
\begin{lstlisting}
Functor f =>
b -> f a -> f b
\end{lstlisting}
\end{block}
\textbf{I already know what this function does}
\end{frame}

\begin{frame}[fragile]
\frametitle{Parametricity}
\begin{block}{?}
\begin{lstlisting}
Applicative f =>
b -> f a -> f b
\end{lstlisting}
\end{block}
What about now?
\end{frame}

\begin{frame}[fragile]
\frametitle{Parametricity}
\begin{center}
There are some limitations of this reasoning
\end{center}
\end{frame}
