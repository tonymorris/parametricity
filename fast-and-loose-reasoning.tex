\begin{frame}
\frametitle{What is Parametricity}
\begin{block}{Danielsson, Hughes, Jansson \& Gibbons \cite{danielsson2006fast} tell us:}
\begin{quotation}
Functional programmers often reason about programs as if
they were written in a total language, expecting the results
to carry over to non-total (partial) languages. We justify
such reasoning.
\end{quotation}
\end{block}
\end{frame}

\begin{frame}[fragile]
\frametitle{Fast and Loose Reasoning}
\begin{lstlisting}
boolean even(int i) =
  ...
\end{lstlisting}
We casually say, ``This function returns one of two things.''
\end{frame}

\begin{frame}[fragile]
\frametitle{Fast and Loose Reasoning}
\begin{lstlisting}
boolean even(int i) =
  even(i)
\end{lstlisting}
and we can discard this third possibility in our reasoning.
\end{frame}

\begin{frame}[fragile]
\frametitle{Fast and Loose Reasoning}
\begin{center}
We are \emph{fast and loose reasoning}.
\end{center}
\end{frame}

\begin{frame}[fragile]
\frametitle{Fast and Loose Reasoning}
\begin{block}{many programming environments involve some, or all of}
\begin{itemize}
  \item \lstinline{null}
  \item exceptions
  \item Type-casing
  \item Type-casting
  \item Side-effects
  \item universal \lstinline{equals}/\lstinline{toString}
\end{itemize}
\end{block}
For similar reasons, these can all be \emph{morally} discarded.
\end{frame}
