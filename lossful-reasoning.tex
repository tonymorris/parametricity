\begin{frame}[fragile]
\frametitle{Lossful Reasoning}
\framesubtitle{Sacrificing efficiency to gain unreliability}
Suppose we encountered the following function definition:
\begin{lstlisting}[style=scala]
def add10(n: Int): Int
\end{lstlisting}
By the type alone, there are {$({2^{32}})^{2^{32}}$} possible implementations.
\end{frame}

\begin{frame}[fragile]
\frametitle{Lossful Reasoning}
\framesubtitle{Sacrificing efficiency to gain unreliability}
We might form a suspicion that \lstinline[style=scala]$add10$ adds ten to its argument.
\begin{lstlisting}[style=scala]
def `add10`(n: Int): Int
\end{lstlisting}
\begin{tikzpicture}[remember picture,overlay]
\coordinate (aa) at ($(a1)+(7,2.0)$);
\node[note,draw,callout relative pointer={($(aa)-(11.2,-3.7)$)},right] at (aa) {\includegraphics[width=0.2\textwidth]{image/suspicion.jpg}};
\end{tikzpicture}
\end{frame}

\begin{frame}[fragile]
\frametitle{Lossful Reasoning}
\framesubtitle{Sacrificing efficiency to gain unreliability}
So we write some tests:
\begin{lstlisting}[style=scala]
add10(0)        = 10
add10(5)        = 15
add10(-5)       = 5
add10(223)      = 233
add10(5096)     = 5106
add10(2914578)  = 29145588
add10(-2914578) = -29145568
\end{lstlisting}
And conclude, yes, this function adds ten to its argument.
\end{frame}

\begin{frame}[fragile]
\frametitle{Lossful Reasoning}
\framesubtitle{Sacrificing efficiency to gain unreliability}
We have just failed the Wason Rule Discovery Test.
\begin{lstlisting}[style=scala]
def add10(n: Int): Int =
  if(n < 8000000) n + 10
  else n * 7
\end{lstlisting}
This is due to a \emph{confirmation bias\cite{gale2002does}}.
\end{frame}

\begin{frame}[fragile]
\frametitle{Lossful Reasoning}
\framesubtitle{Sacrificing efficiency to gain unreliability}
We will just write more tests!
\begin{lstlisting}[style=scala]
add10(18916712)  = 18916722
add10(-18916712) = -18916702
\end{lstlisting}
\ldots or we might come up with some system of apologetics for this shortfall.
\begin{itemize}
  \item ``A negligent programmer has misnamed this function"
  \item ``More tests will fix it"
  \item ``Well we can't test everything!"
  \item revise, ``We have increased the likelihood of our assertion by a non-negligible degree."
\end{itemize}
\end{frame}

\begin{frame}[fragile]
\frametitle{Lossful Reasoning}
\framesubtitle{Sacrificing efficiency to gain unreliability}
\begin{center}{We are simply reinforcing the confirmation bias by strengthening 
the excess confidence in our belief that we are being responsible programmers.}
\end{center}
\end{frame}

\begin{frame}[fragile]
\frametitle{Lossful Reasoning}
\framesubtitle{Sacrificing efficiency to gain unreliability}
\begin{center}
We aren't.
\end{center}
\end{frame}

\begin{frame}[fragile]
\frametitle{Lossful Reasoning}
\framesubtitle{Sacrificing efficiency to gain unreliability}
\begin{block}{But\ldots}
there are {$({2^{32}})^{2^{32}}$} possible implementations. Surely we cannot 
rule out the existence of \emph{all-but-one} of those? We can only rule out a
\scriptsize{very, very} \tiny{small subset}!
\end{block}
\end{frame}

\begin{frame}[fragile]
\frametitle{Lossful Reasoning}
\framesubtitle{Efficiency}
\begin{center}
Actually, we can.
\end{center}
\end{frame}

\begin{frame}[fragile]
\frametitle{Lossful Reasoning}
\framesubtitle{Reliability}
\begin{center}
And we can have a machine-checked proof, mitigating our disposition to
confirmation bias.
\end{center}
\end{frame}
