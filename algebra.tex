\begin{frame}
\begin{block}{Types can tell us about inhabitants}
Algebraic Data Types can be thought of in terms of regular algebraic equations
\end{block}
\end{frame}

\begin{frame}
\begin{block}{Some examples include}
\begin{itemize}
  \item \textbf{sum types}

        \lstinline{Either A B} or ``A or B'' corresponds to the equation \lstinline{A + B}
  \item \textbf{product types}

        \lstinline{(A, B)} or ``A and B'' corresponds to the equation \lstinline{A * B}
  \item \textbf{exponentiation}

        \lstinline{A -> B} corresponds to the equation \lstinline{B}\textsuperscript{\lstinline{A}}
  \item \textbf{unit}

        given \lstinline{data Unit = Unit},

        the \lstinline{Unit} data type corresponds to the value \lstinline{1}
  \item \textbf{void}

        given \lstinline{data Void},

        the \lstinline{Void} data type corresponds to the value \lstinline{0}
\end{itemize}
\end{block}
\end{frame}

\begin{frame}[fragile]
\begin{block}{Let's look at \lstinline{Bool}}
\begin{lstlisting}
data Bool = True | False
\end{lstlisting}
\begin{itemize}
  \item<1-> The \lstinline{True} constructor has no arguments, which is equivalent to carrying \lstinline{Unit}
  \item<1-> The \lstinline{False} constructor has no arguments, which is equivalent to carrying \lstinline{Unit}
  \item<2-> The whole data type carries \lstinline{1} or \lstinline{1}
  \item<2-> \lstinline{Bool ~ 1 + 1 ~ 2}
\end{itemize}
\end{block}
\end{frame}

\begin{frame}[fragile]
\begin{block}{How about \lstinline{Maybe a}}
\begin{lstlisting}
data Maybe a = Nothing | Just a
\end{lstlisting}
\begin{itemize}
  \item<1-> The \lstinline{Nothing} constructor has no arguments, which is equivalent to carrying \lstinline{Unit}
  \item<1-> The \lstinline{Just} constructor has an argument \lstinline{a}
  \item<2-> The whole data type carries \lstinline{1} or \lstinline{a}
  \item<2-> \lstinline{Maybe a ~ 1 + a}
\end{itemize}
\end{block}
\end{frame}

\begin{frame}[fragile]
\begin{block}{Another one \lstinline{Either Void a}}
\begin{lstlisting}
\end{lstlisting}
\begin{itemize}
  \item<1-> The \lstinline{Left} constructor carries \lstinline{0}
  \item<1-> The \lstinline{Right} constructor has an argument \lstinline{a}
  \item<2-> The whole data type carries \lstinline{0} or \lstinline{a}
  \item<2-> \lstinline{Either Void a ~ 0 + a ~ a}
\end{itemize}
\end{block}
\end{frame}

\begin{frame}[fragile]
\begin{block}{and another \lstinline{(Void, a)}}
\begin{lstlisting}
\end{lstlisting}
\begin{itemize}
  \item<1-> The whole data type carries \lstinline{0} and \lstinline{a}
  \item<2-> \lstinline{(Void, a) ~ 0 * a ~ 0}
\end{itemize}
\end{block}
\end{frame}

\begin{frame}
\begin{block}{lost of \lstinline{Bool}}
\begin{itemize}
  \item<1-> \lstinline{(Bool, Bool)}
  \item<1-> \lstinline{Either Bool Bool}
  \item<1-> \lstinline{Bool -> Bool}
  \item<2-> \lstinline{2 * 2}
  \item<2-> \lstinline{2 + 2}
  \item<2-> \lstinline{2}\textsuperscript{\lstinline{2}}
  \item<3-> These are all \lstinline{4}
\end{itemize}
\end{block}
\end{frame}

\begin{frame}
\begin{block}{Inhabitants}
The resulting algebraic equation gives us the number of \emph{inhabitants}.

Or, the number of values with that type.
\end{block}
\end{frame}

\begin{frame}
\begin{block}{Inhabitants}
\begin{itemize}
  \item<1-> \lstinline{Maybe (Bool -> Maybe Bool)}
  \item<1-> has \lstinline{1 + 2}\textsuperscript{\lstinline{(1 + 2)}} inhabitants
  \item<1-> \lstinline{9} inhabitants
  \item<2-> \scriptsize{\lstinline{(Either Bool (Maybe Bool), Bool, (Unit, Bool), Either Void Bool)}}
  \item<2-> has \lstinline{(2 + (1 + 2)) * 2 * (1 * 2) * (0 + 2)} inhabitants
  \item<2-> \lstinline{40}
\end{itemize}
\end{block}
\end{frame}

\begin{frame}
\begin{block}{Algebraically}
\begin{center}
What is \lstinline{[a]}?
\end{center}
\end{block}
\end{frame}

\begin{frame}
\begin{block}{Lists}
\begin{itemize}
  \item<1-> \lstinline{[a]} is either zero \lstinline{a} or one \lstinline{a} or two \lstinline{a} \ldots
  \item<2-> \lstinline{a}\textsuperscript{\lstinline{0}} \lstinline{+ a}\textsuperscript{\lstinline{1}} \lstinline{+ a}\textsuperscript{\lstinline{2}} \ldots
  \item<3-> using algebraic rules, this simplifies to \lstinline{1 + a * [a]}
  \item<3-> \lstinline{1} or \lstinline{(a and [a])}
  \item<3-> The \lstinline{[]} \emph{(carrying \lstinline{Unit})} or \lstinline{(:)} constructor
\end{itemize}
\end{block}
\end{frame}
