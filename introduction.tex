\begin{frame}
\begin{center}
Let us remind ourselves.
\end{center}
\begin{center}
What is Functional Programming? What does it \emph{mean}?
\end{center}
\end{frame}

\begin{frame}[fragile]
\begin{block}{Suppose the following program \ldots}
\begin{lstlisting}[style=java]
int wibble(int a, int b) {
  counter = counter + 1;
  return (a + b) * 2;
}

/* arbitrary code */

blobble(wibble(x, y), wibble(x, y));
\end{lstlisting}
\end{block}
\end{frame}

\begin{frame}[fragile]
\begin{block}{and we refactor out these common expressions \ldots}
\begin{lstlisting}[style=java]
int wibble(int a, int b) {
  counter = counter + 1;
  return (a + b) * 2;
}

/* arbitrary code */

blobble(`wibble(x, y)`, `wibble(x, y)`);
\end{lstlisting}
\end{block}
\end{frame}

\begin{frame}[fragile]
\begin{block}{assign the expression to a value}
\begin{lstlisting}[style=java]
int wibble(int a, int b) {
  counter = counter + 1;
  return (a + b) * 2;
}

int r = `wibble(x, y);`

/* arbitrary code */

blobble(`r`, `r`);
\end{lstlisting}
\end{block}
\end{frame}

\begin{frame}[fragile]
\begin{center}
Did the program just change?
\end{center}
\end{frame}

\begin{frame}[fragile]
\begin{block}{Yes, the program changed \ldots}
\begin{lstlisting}[style=java]
int wibble(int a, int b) {
  `counter = counter + 1;`
  return (a + b) * 2;
}

int r = wibble(x, y);

/* arbitrary code */

blobble(r, r);
\end{lstlisting}
\end{block}
\end{frame}

\begin{frame}[fragile]
\begin{block}{Suppose this slightly different program \ldots}
\begin{lstlisting}[style=java]
int pibble(int a, int b) {
  return (a + b) * 2;
}

/* arbitrary code */

globble(pibble(x, y), pibble(x, y));
\end{lstlisting}
\end{block}
\end{frame}

\begin{frame}[fragile]
\begin{block}{and we refactor out these common expressions \ldots}
\begin{lstlisting}[style=java]
int pibble(int a, int b) {
  return (a + b) * 2;
}

/* arbitrary code */

globble(`pibble(x, y)`, `pibble(x, y)`);
\end{lstlisting}
\end{block}
\end{frame}

\begin{frame}[fragile]
\begin{block}{assign the expression to a value}
\begin{lstlisting}[style=java]
int pibble(int a, int b) {
  return (a + b) * 2;
}

int r = `pibble(x, y);`

/* arbitrary code */

globble(`r`, `r`);
\end{lstlisting}
\end{block}
\end{frame}

\begin{frame}
\begin{center}
This time, did the program just change?
\end{center}
\end{frame}

\begin{frame}
\begin{block}{It's the same program}
For given inputs, the same outputs are given, with no observable changes to the program
\end{block}
\end{frame}

\begin{frame}
\begin{block}{Functional Programming is the idea that}
We can \textbf{always replace expressions with a value, without affecting the program behaviour}
\end{block}
\begin{center}
\tiny{This property of expressions is called \emph{referential transparency}.}
\end{center}
\end{frame}
\begin{frame}
\begin{block}{Consequences}
A commitment to Functional Programming has many consequences.
\end{block}
\end{frame}

\begin{frame}
\begin{center}
One of the most important of those is
\end{center}
\begin{center}
Parametricity
\end{center}
\end{frame}
