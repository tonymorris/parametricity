\begin{frame}[fragile]
\frametitle{The Parametricity Trick}
\begin{block}{This will only work if\ldots}
\begin{itemize}
\item<1-> you write computer programs with inveterate exploitation of the functional programming thesis
\item<1-> you understand that anything else is \textbf{completely insane}
\item<2-> and if you don't, you're just being a wrong person
\end{itemize}
\end{block}
\end{frame}

\begin{frame}[fragile]
\frametitle{Reminder}
\begin{block}{So what is functional programming?}
\begin{itemize}
\item<1-> a means of programming by which expressions are \emph{referentially transparent}.
\item<2-> but what is referential transparency?
\end{itemize}
\end{block}
\end{frame}

\begin{frame}
\frametitle{Referential Transparency}
\begin{itemize}
  \item<1> referential transparency is a potential property of expressions
  \item<2> functions provide users with referentially transparent expressions
\end{itemize}
\begin{block}<3>{The Test for Referential Transparency}
An expression \lstinline$expr$ is referentially transparent if in all programs \lstinline$p$, all occurrences of \lstinline$expr$ in \lstinline$p$ can be replaced by the result assigned to \lstinline$expr$ without causing an observable effect on \lstinline$p$.
\end{block}
\end{frame}

\begin{frame}[fragile]
\frametitle{Referential Transparency}
\begin{block}<1>{Example program}
\begin{lstlisting}
p = {
  result = expr
  result = expr
  f(expr, expr)
}
\end{lstlisting}
\end{block}
\begin{block}<2>{Refactoring of program}
\begin{lstlisting}
p = {
  f(result, result)
}
\end{lstlisting}
\end{block}
\begin{block}<3>{}
Is the program refactoring observable for all values of \lstinline$f$?
\end{block}
\end{frame}

\begin{frame}[fragile]
\frametitle{Functional Programming}
\begin{center}
FP is a commitment to preserving referential transparency
\end{center}
\end{frame}
